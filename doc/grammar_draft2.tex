\documentclass{memarticle}
\usepackage{marvosym} % to define \MVAt
\newcommand{\docnumber}%
  {draft 2}

\newcommand{\doctitle}%
  {DRAFT---{\color{magenta}Specification of the Fermilab Hierarchical Configuration Language}---DRAFT}

\newcommand{\shorttitle}%
 {DRAFT---Specification of the FHiCL---DRAFT}

\newcommand{\authors}%
  {
  Ryan Putz
  } 

\hypersetup%
  { pdfauthor={}%
  , pdftitle={\doctitle}%
  , pdfkeywords={}%
  , pdfsubject={}%
  }

\tightlists

\makeindex

\begin{document}
\topmatter
\setlength{\parindent}{0in}
% 	Title stuff
\title{Grammar Specification}
\author{Jim~Kowalkowski \\ 
	Marc~Paterno}
\maketitle

%	Body of document

\section{Introduction}
	
	\subsection{Purpose}
	{
		This document provides the formal specification
		for the \emph{Fermilab Hierarchical Configuration Language}, FHiCL.
		This specification includes several aspects of FHiCL:
		\begin{itemize}
			\item User Requirements 
			\item Definitions of Structural Elements
			\item Canonical Value Representations
			\item Description of Valid Syntactical Grammar
		\end{itemize}
	}

	\subsection{Rationale}
	{
		FHiCL was developed in order to produce
		a standard language for the storage,
		communication, 
		and manipulation
		of scientific parameter sets.
  		%Why does this project exist?
	}
	
	\subsection{Scope of This Facility}
	{
		A common storage language was needed 
		in order to produce language bindings 
		for the various programming languages in use 
		at this facility.
		Having these bindings would allow parameter sets
		stored using FHiCL
		to be loaded 
		and manipulated in multiple programming environments.
  		%What does this include? What does it not include?
	}
	
\section{User Requirements}

	Representatives from LQCD, mu2e, NOvA, and JDEM contributed to the requirements 
	of this language.

	\begin{itemize}
	\item allowable special characters within string values are newline, tab
	\item values that are \emph{simple words} and without whitespace do not need to be quoted
	\item single quoted values are read verbatim (no escape processing)
	\item value that are double quoted can contained escaped special characters, double quotes and backslashes
	\item two tables are the same when their hash code is the same (the byte sequences fed into the hash must be identical)
	\item the canonical representation of an atom is a sequence of printable characters
	\item values of true and ``true'' are identical
	\item every atom can be requested in canonical string form
	\item to include leading or trailing zeros in a any number, the number must be quoted
	\item the small range of a real or integer value is 1,000,000
	\item real number with no fraction will be converted to integer format if within the small range
	\item a canonical real has no leading zeros in exponent or fraction, lower case e, with plus or minus
	\item canonical integers have no leading zeros
	\item null is not supported
	\item infinity, +infinity becomes ``infinity'', 
	\item ``infinity'' stays ``infinity''
	\item -infinity is support
	\item a leading $+$ is not legal,
	\item a leading 0 is not legal unless it is the sole character.
	\item adding a double to a parameter set programmatically will have a rule that specified how it will be done
	\item 00.000E+000 will be ``0.0'' in canonical form
	\item any exponent as e+0 will be stripped in the canonical form
	\item -0.0 retains the negative
	\item nil and ``nil'' are the same thing
	\item string concatenation operations are permitted, but only quoted string values
	\item no unquoted white space is permitted
	\item quotes for string values can be left out if the string value has no white space and is ``simple''
	\end{itemize}
	
\section{FHiCL Syntax}

	The FHiCL syntax is defined by the following bison grammar:
	\begin{verbatim}
		\import{"bnf.y"}
	\end{verbatim}

	In this grammar,
	all uppercase names denote tokens.
	These tokens are defined by the following flex specification:
	%true and false are both tokens, and yet are all lowercase characters
	\begin{verbatim}
	\include{"bnf.l"}
	\end{verbatim}
	\subsection{Preprocssor Directives}
		\subsubsection{Includes}
			In order to combine multiple files into a single resulting document,
			an \emph{include} statement is used to tell which file's values should be added to the document.
			A FHiCL \#include statement differs from the C++ \#include statement 
			in that the FHiCL \#include acts 
			more asa union of two documents
			, as opposed to just allowing one file to access another.
			The \emph{include} statement syntax is as follows:
			\begin{verbatim}
				#include "filename.ext"
			\end{verbatim}
			\vspace{1mm}
			\par
			Where the quoted string "filename.ext" represents the file name and file extension of the included file.
			\vspace{1mm}
			\par
			\bf NOTE: \rm There is exactly one space between '\#include' and 'filename.ext'.
			Also, the filname must be enclosed in double quotes.
			If there is multiple spaces and/or no double quotes, 
			the include statement will be treated as a comment.
			\par
			\bf NOTE: \rm Circular includes 
			and repetitive includes 
			are \emph{not} supported and should be checked for by the parser.
			\par
			\bf NOTE: \rm Included values can override existing values 
			if both definition's names match 
			and both values are within the same scope.	

		\subsubsection{References}
			In order to assign a pre-existing element 
			as a value for another element
			the use of the FHiCL \emph{reference} notation is required:
			\begin{verbatim}
				@local::
				
				x : 5
				y : @local::x
				z : @db::x
			\end{verbatim}
			
	\subsection{Low-Level Entities}
		\bf Note: \rm For all rules in this section,
		whitespace is not allowed between tokens.
		\subsubsection{Char}
			A \emph{char} is one of:
			\begin{enumerate}
				\item any ASCII character except for:
				\begin{itemize}
					\item double-quote ('') 
					\item reverse solidus (\textbackslash)
					\item control characters
				\end{itemize}
				\item (\emph{printable} characters)
				\item one of a number escape sequences, noted below:
				\begin{itemize}
					\item escaped double-quote (\textbackslash")
					\item reverse solidus (\textbackslash\textbackslash)
					\item solidus (\textbackslash/)
					%It says "noted below", but where?
				\end{itemize}
				There are a number of reserved char values:
				\begin{itemize}
					\item colon ( : )
					\item double colon ( :: )
					\item equals ( = )
					\item left/right brace ( \{\} )
					\item left/right bracket ( [] )
					\item left/right paren ( () )
					\item at sign ( @ )
				\end{itemize}		
			\end{enumerate}
			
		\subsubsection{Atoms}
%			\bf Note: \rm The definitions of \emph{atom},
%			\emph{table}, 
%			and \emph{sequence} are mutually interdependent.
%			\vspace{1mm}
%			\par
			The most basic unit of FHiCL is the \emph{atom},
			which is defined as: 
			\begin{verbatim}
				atom: number | string | NIL | BOOL_TOK | REF
			\end{verbatim}
			\vspace{1mm}
			EBNF:
			\begin{verbatim}
				atom   =>    char | string
				string =>    alpha[alnum]* | digit[alnum]*
			\end{verbatim}
			
	\subsection{Mid-level Entities}
		\bf Note: \rm For all rules in this section,
		whitespace is allowed only where specified by the whitespace token \emph{ws}.
		\subsubsection{Comments}
			FHiCL comments are denoted by the \emph{\#} symbol,
			which is placed at the beginning of the comment.
			FHiCL comments are single-line,
			and should be ignored by parsers.
		\subsubsection{Names}
			A \emph{name} is used to begin a \emph{definition}.
			\vspace{1mm}
			\par\bf{Example:}
			\rm
			\begin{verbatim}
				x: 1.0
			\end{verbatim}
			\vspace{1mm}
			In this case, 
			"x" is a \emph{name} 
			used to begin the \emph{definition} x: 1.0
		\subsubsection{Hierarchical Names}
			A hierarchical name,
			or \emph{hname} is a compound name 
			using the \emph{dot index} to denote levels of a container.
			%is used in begin an \emph{override}:
			\begin{verbatim}
				tab1:{x: 1.0, y: 2.0, z: 3.0}
				tab1.x = 5
			\end{verbatim}
			\vspace{1mm}
			\par
			EBNF:
			\begin{verbatim}
				hname => atom DOT_INDEX atom
			\end{verbatim}
			\vspace{1mm}
			\par
			Here we have a table, named \emph{tab1} with a member \emph{x}, \emph{y}, and \emph{z}. 
			In order to access the \emph{x} member of \emph{tab1}
			we must use a \emph{dot index} 
			(denoted by a ".") 
			to access the \emph{x} member.
			\break
			\break
			The name "tab1.x"
			is an \emph{hname}.
	
		\subsubsection{Values}
			An element of type \emph{value} is either a single atom, 
			a collection of atoms,
			or a collection of definitions.
			%An element of type \emph{value} is either a:
			%\begin{itemize}
			%	\item \emph{table},
			%	\item \emph{sequence},
			%	\item or an \emph{atom}:
			%\end{itemize}
			Example:
			\begin{verbatim}
				a : 1.0
				#Where "1.0" is the value of the atom named "a"
			\end{verbatim}
			
			\vspace{1mm}
			\par
			EBNF:
			\begin{verbatim}
				value => table|sequence|atom
			\end{verbatim}	
			\vspace{1mm}
			\par
			\bf Note: \rm see definitions for \emph{table} 
			and \emph{sequence} 
			in the next section
	\subsection{High-Level Entities}

		\bf Note: \rm For all the rules in this section,
		whitespace is allowed between any two tokens,
		and is not significant.
						
		\subsubsection{Definitions}
			An element of type \emph{definition} is used to
			create a new element:
			\begin{verbatim}
				a : 1.0
			\end{verbatim}
			\vspace{1mm}
			\par
			EBNF:
			\begin{verbatim}
				definition => atom COLON value
			\end{verbatim}	
			
		\subsubsection{Overrides}
			An element of type \emph{override} is used to
			change the value of an existing element,
			or to create a new element in a \emph{table} or \emph{sequence}.
			The syntax for an override:
			\begin{verbatim}
				a: 1.0 #Declaration and initialization
				a: 5.0 #Override (Assignment)
				
				OR
				
				tab1:{ a:1 b:2 c:3 }
				tab1.d : 5 #Creating a new element 'd' in table 'tab1'
			\end{verbatim}
			\vspace{1mm}
			EBNF:
			\begin{verbatim}
				override => atom COLON value
			\end{verbatim}	
			\bf Note: \rm the \emph{name} for an override 
			is an \emph{hname}.
		\subsubsection{Tables}
			Elements of type \emph{table} 
			are space- or line-separated collections of definitions 
			and are denoted by (possibly empty) braces:
			\begin{verbatim}
				tab1:{a: 1.0 b: 2.0 c: 3.0}
			\end{verbatim}
			\vspace{1mm}
			EBNF:
			\begin{verbatim}
				table => LBRACE table_body RBRACE
				table_body => | table_items
				table_items => table_item | [table_item + "," + table_items]
				table_item => atom|definition
			\end{verbatim}
			\par
			\bf NOTE:\rm Tables may contain comments 
			\bf IF AND ONLY IF \rm
			the table elements are line-separated. 
			Comments cannot exist inbetween space-separated elements.
		\subsubsection{Sequences}
			Elements of type \emph{sequence} 
			are comma-separated collections of atoms 
			and are denoted by (possibly empty) brackets:
			\begin{verbatim}
				seq1:[a, b, c, d]
			\end{verbatim}
			\vspace{1mm}
			EBNF:
			\begin{verbatim}
				sequence => LBRACKET sequence_body RBRACKET
				sequence_body => | sequence_items
				sequence_items => sequence_item | [sequence_item + "," + sequence_items]
				sequence_item => atom
			\end{verbatim}
			\par
			\bf NOTE: \rm Sequences \bf CANNOT \rm contain comments.
		\subsubsection{The Document}
			The \emph{document} is the highest-level construct 
			in FHiCL.
			Any implementation of a FHiCL parser
			processes a \emph{document}
			as if it were a single string.

			A \emph{document} consists of exactly one,
			possibly empty,
			\emph{table} such as:
			\begin{verbatim}
				#Document start
				main:{a: 1.0, b: "hi", c: dog, a= 5.0}
				#Document end
			\end{verbatim}	
			\vspace{1mm}
			EBNF:
			\begin{verbatim}
				document => table
			\end{verbatim}

\section{FHiCL Semantics}
	\subsection{High-level result of a successful parse}

		The result of parsing a \emph{document}
		is a single \emph{table}.
		The \emph{definition}s and \emph{override}s
		appearing before the top-level \emph{table}
		are intended to allow the user
		to supply values to be substituted into element in the \emph{table}.
		The \emph{definition}s and \emph{override}s
		appearing after the top-level \emph{table}
		are intended to allow the user
		to replace values in that table.

	\subsection{Representation of Atoms\index{atoms}}
		In the parse results,
		all \emph{atom}s
		except for \texttt{nil} and \emph{ref}
		are represented
		as character strings.
		The atom \texttt{nil} is represented by a 
		value specified by the binding for a given programming language.
		The resolution of \emph{ref}s is described in section~\ref{sec:refs} below.
		\vspace{1mm}
		\par
		Each language binding
		provides its own mechanism
		for turning atoms of type \emph{integer}, \emph{real} and \emph{complex}
		from their string representation
		into the appropriate numerical representation.

	\subsection{Resolution of \emph{Ref}s\label{sec:refs}}
		Atoms of type \emph{ref} are replaced
		by the value indicated by the \emph{hname} part of the \emph{ref},
		where the environment in which the \emph{hname} is evaluated is determined
		by the \texttt{db} or \texttt{local} at the end of the \emph{ref}.
		\vspace{1mm}
		\par
		The presence of \texttt{local} indicates 
		that the scope in which the \emph{hname} is to be sought
		is the previously-read \emph{document} text.
		The presence of 
		\texttt{db} indicates
		that the scope in which the \emph{hname} is evaluated
		is the single database
		to which the parser has access.
		\vspace{1mm}
		\par
		If the parser has no access to a database,
		and a \emph{ref} which ends in \texttt{db} is encountered,
		a parse failure results.
		If,
		in the appropriate scope,
		the \emph{hname} in a \emph{ref} does not resolve to any \emph{value},
		a parse failure results.
		
\section{Features of Programming Language Binding}

	\subsection{Output}

	\subsection{Storage}
		
		
\section{General Requirements}

	\subsection{Additional Requirements for Dynamically Typed Languages}

		Tables and sequences should be represented by a built-in type of the
		programming language.

		If the target programming language has a standard JSON library, we
		want to make sure that our constructs can be translated to JSON format
		and back without use of any FHiCL-specific library.

		It is important that code that uses the representation of a
		\emph{table} not need any FHiCL-specific code.

\section{Output Requirements}

	\subsection{Output Intended for Human Reading}
		%\begin{fixme}
  		Pretty-printing description here.
		%\end{fixme}
	\subsection{Output Intended for Machine Reading}
		%\begin{fixme}
  		ASCII dump facility, machine reparsable, and the same on all
  		platforms. Very terse.
		%\end{fixme}

\section{Glossary}
		\subsubsection{Alphas}
			An \emph{alpha} is any of the ASCII characters a-z or A-Z.
		\subsubsection{Digits}
			A \emph{digit} is any of the ASCII characters 0-9.
		\subsubsection{White Space}
			A \emph{ws} is one of the three whitespace characters.
		\subsubsection{Alphanumerics}
			An \emph{alnum} is any of the ASCII characters a-z, A-Z, 0-9 or other \emph{printable} characters

	%\appendix

	%\printindex

\end{document}

\end{document}

