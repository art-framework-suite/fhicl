\documentclass{memarticle}

\usepackage{marvosym}%          to define \MVAt

\newcommand{\fhicl}%
 {FHiCL\xspace}

\newcommand{\shorttitle}%
 {\fhicl Quick Start Guide}

\newcommand{\docnumber}{\shorttitle}

\newcommand{\doctitle}{Quick Start Guide for \fhicl:\\
the Fermilab Hierarchical Configuration Language}

\date{\today}

\newcommand{\authors}{The ART Development Team:\\
  Walter E. Brown%
, Chris Green%
, Jim Kowalkowski%
, Marc Paterno%
, Ryan Putz%
}

\firmlists%\tightlists

\begin{document}
\flushbottom
\raggedbottomsectiontrue
\thispagestyle{empty}
\pagenumbering{arabic}

\begin{center}
  {\LARGE\doctitle}
  { \vskip\baselineskip }
  { \authors }
\end{center}
\noindentcolorrule{Red}
\tableofcontents*
\noindentcolorrule{Red}


\chapter{Introduction}

The purpose of this document
is to explain and demonstrate
the major syntax and semantics
of the Fermilab Hierarchical Configuration Language, \fhicl%
\footnote{%
  The customary pronounciation is /fIkl/.%
},
so that users may become comfortable
with its features and intended use.

The end product of many years
of experience with other configuration languages and notation,
\fhicl has been carefully designed
to allow its users to express, record, and retrieve
sets of software parameters
used to \term{configure}
(prepare for a particular purpose)
a program's execution.
We will use the \term{parameter set}
to denote any specific collection
of named values
accessible to a user's program
while it is running;
a \fhicl \term{document}
is a textual representation
of such a parameter set.
Two or more \fhicl documents
are said to be \term{equivalent}
if they lead to identical parameter sets.

Depending on the programming language
used to write the user program,
one of several software subsystems is used
to produce a parameter set
from a given \fhicl document.
Each such subsystem is known as a \term{binding}
of the \fhicl specification
to the programming language,
and provides for the analysis and interpretation
of \fhicl documents.

While it is up to each binding
to specify how users interface
to the parameter sets produced by that binding,
it is fundamental that
users be able to query a parameter set
by providing it a name
(in the form of a string)
in order to obtain the value corresponding to that name.
Further,
a parameter set
must be capable of producing a document
that is equivalent
to the original document
that gave rise to that parameter set.


\chapter{Documents}

\section{Text}

A \fhicl document is simply a sequence of characters
(\ie text),
structured as described below,
and is commonly stored in a file
whose name is conventionally suffixed \fclcode{.fcl}.
For example,
\fclcode{my_config.fcl} might be the name of such a file.
Any conventional text editor
(\eg emacs, vi, nedit, \ldots)
may be used to create or to update a \fhicl file.%
\footnote{%
  Note that \fhicl documents
  need not be represented via any file.
  A binding may, for example,
  obtain a \fhicl document
  from a database,
  via a string in a conventional programming language,
  or
  via any other mechanism that can denote simple text.
  A binding is free
  to support as many document sources
  as it wishes.%
}

\section{Name-value pairs}

A document consists principally
of \term{name-value pairs}.%
\footnote{%
  A name-value pair is sometimes known
  as an \term{association},
  because parameter set lookup
  is designed to take a name
  and retrieve the associated value.
  In this Guide
  we will usually prefer the simpler \term{pair} nomenclature.%
}
There may be as many or as few such pairs
as desired.%
\footnote{%
  A document consisting of no name-value pairs
  is said to be \term{empty},
  as is the parameter set
  that a binding would generate
  from such a document.%
}
In each pair,
a colon (\verb|:|) separates the \term{name}
from the corresponding \term{value}.

At least one blank, tab, or newline character
(collectively known as \term{whitespace})
must separate one pair from the next.
The following document,
consisting of three name-value pairs,
uses the minimum required whitespace:
\Needspace{.17in}
\begin{fcllisting}[texcl,escapechar=`]
n:1 pi:3.14159 label:"horizontal axis"
\end{fcllisting}

\section{Optional whitespace}

Within a \fhicl document,
additional whitespace
may be used
at the discretion of the author.%
\footnote{%
  Such optional whitespace
  is commonly used
  to produce indentation or alignment.%
}
A binding will ignore
any such optional whitespace
while producing a parameter set
from the document.

The following document
is equivalent to the single-line document shown above.
Consisting of the same three name-value pairs,
this variation employs extra whitespace
to improve readability
by (a) placing each pair on an individual line
and (b) aligning the values:
\Needspace{0.5in}
\begin{fcllisting}[texcl,escapechar=`]
n     : 1
pi    : 3.14159
label : "horizontal axis"
\end{fcllisting}

\section{Preview of other features}

In addition to name-value pairs,
a \fhicl document may contain
\term{comments} (commonly also known as \term{remarks} or \term{annotations})
as well as \term{prolog} sections.
Further,
a document's contents may be spread over several files
via the \fhicl \term{include} mechanism.
Each of these features will be introduced
in this Guide.
We will also provide more details
about \fhicl{}'s names and values.


\chapter{Comments}

Document providers often wish
to annotate the document's contents.
For example,
it is common to provide a provenance
giving the original author's name and date,
followed by a revision history.
Other annotations include
introductory overviews for each section
of a document,
or even brief descriptions
of individual name-value pairs.
\fhicl comments can serve all of these purposes,
and more.

\fhicl provides two ways to \term{introduce} (start) a comment:
\begin{itemize}
  \item With a single \fclcode{#} character,%
          \footnote{%
          This notation is adopted
          from such scripting languages
          as bash, perl, and python.
          The introductory \fclcode{#} character
          is known variously as a
          \term{pound sign},
          \term{hash mark},
          \term{sharp},
          or \term{octothorpe}%
          .}
  or else
  \item With two consecutive forward slashes (\fclcode{//}).\footnote{%
          This notation is adopted
          from such programming languages
          as BCPL and \cpp{}%
          .}
\end{itemize}
Either or both of these comment introductions
may be used within any \fhicl document.

The remainder of the line
is the \term{body} of the comment,
and provides whatever information
the author may desire.
The comment implicitly terminates
at the end of the line,
although an author may choose to continue his annotation
onto any number of additional comments
on subsequent lines.

The following document illustrates
the various forms
that a \fhicl comment may take.
\Needspace{0.67in}
\begin{fcllisting}[texcl,escapechar=`]
# This is a comment
// This is also a comment
foo  : "bar"   # this is a comment "in the margin" ...
foo2 : "bar2" // ... and so is this
\end{fcllisting}


\chapter{Names}

\section{Spelling}

The spelling rules for \fhicl names
match the spelling rules
for identifiers in many programming languages:
\begin{itemize}
  \item Each name begins with a letter
        or with a \fclcode{_} (underscore) character.
  \item The name may be spelled
        with as many additional consecutive
        letters, underscores, or digits as desired.
        No other characters
        (\eg punctuation or whitespace)
        may be embedded within a name.
  \item Capitalization matters:
        the name \fclcode{X} is not the same name
        as the uncapitalized name \fclcode{x}.
        Similarly,
        \fhicl treats the names
        \fclcode{Hello}, \fclcode{HELLO}, and \fclcode{hello}
        as three distinct, unrelated names.
\end{itemize}%

\section{Name reuse}

Consider the following document,
noting especially the reuse of the name \fclcode{a}.
When a binding processes this document,
how many name-value pairs
will the resulting parameter set contain?
\Needspace{0.5in}
\begin{fcllisting}[texcl,escapechar=`]
a : 1
b : 2
a : 3
\end{fcllisting}
The answer is two.
\fhicl provides that,
if a document has two name-value pairs
that have a name in common,
the value in the later pair
\term{overrides} (supercedes) the earlier one.\footnote{%
  If more than two pairs
  have a name in common,
  the second pair overrides the first
  as described above
  until the third such pair is encountered.
  Then the third pair overrides the second
  until a fourth pair is encountered,
  and so on.
  In this way,
  the last pair using that name
  will ultimately override all the earlier ones
  with the same name.%
}
Therefore, in the above example,
when \fclcode{a} is looked up in the parameter set,
the associated value will be found to be \fclcode{3}.

\section{References}

Within some documents,
a common scenario
is to require that
two (or more) name-value pairs
with distinct names
nonetheless provide the same value.
The most obvious approach
is simply to duplicate the value
in each pair:
\Needspace{0.34in}
\begin{fcllisting}[texcl,escapechar=`]
m : 1
n : 1
\end{fcllisting}
However,
as is true when writing programs,
such duplication becomes problematic over time
because it is not obvious by inspection
that these two values
are intended always to be identical.
As a result,
if a future update
were to change the value associated with \fclcode{m},
the required corresponding change to \fclcode{n}
could be easily overlooked,
especially if there were a great many intervening lines.

To help avoid such an unhappy scenario,
\fhicl allows a value to refer
to a previously-provided value:
\Needspace{0.34in}
\begin{fcllisting}[texcl,escapechar=`]
m : 1
n : @local::m
\end{fcllisting}
A construction such as \fclcode{@local::m}
is known as a \fhicl \term{reference}.
Each reference consists of the prefix \fclcode{@local::}
followed by a name (here, \fclcode{m})
from an earlier name-value pair.%
\footnote{%
  The binding will report an error during processing
  if the name hasn't yet been seen.%
}

When a binding processes such a reference,
the name is looked up
among the name-value pairs processed so far,
and the corresponding value
\term{substituted} (used in place of the reference).
As a result,
the two pairs can no longer get out of sync,
as any change to the value of the first
will automatically be propagated to the second.


\chapter{Values}

\section{Classifications}

At a high level,
each \fhicl value can be classified
as either \term{atomic} or \term{structured}.
An \term{atom} (atomic value)
is either
  a \term{nil},
  a \term{boolean},
  a \term{number},
or
  a \term{string},
while a structured value is either
  a \term{complex} number,
  a \term{sequence},
or
  a \term{table}.
Each of these is further described in the following sections.

\section{Nil value}

The literal \fclcode{nil}
serves as a placeholder value,
distinct from all other \fhicl values.
It is suitable for constructing a name-value pair
when no other \fhicl value will do.
\Needspace{0.17in}
\begin{fcllisting}[texcl,escapechar=`]
higgs : nil
\end{fcllisting}

\section{Boolean values}

The literals \fclcode{true} and \fclcode{false}
correspond to the customary truth values.
\Needspace{0.17in}
\begin{fcllisting}[texcl,escapechar=`]
debug : true
\end{fcllisting}

\section{Numeric values}

\subsection{Form}

As in most programming languages,
a \fhicl number
can have up to four parts:
\begin{itemize}
  \item The \term{sign} part consists of
        a single \fclcode{+} or \fclcode{-} character,
  \item The \term{whole} part consists of
        a non-empty sequence of digits,
        such as \fclcode{0} or \fclcode{123};
        any superfluous leading zeroes
        will be ignored by the binding.
  \item The \term{fraction} part consists of
        a single \fclcode{.} character
        followed by
        a possibly empty sequence of digits.
  \item The \term{exponent} part consists of
        a single \fclcode{e} or \fclcode{E} character,
        optionally followed by a sign,
        followed by
        a non-empty sequence of digits.
        Examples include \fclcode{E5} and \fclcode{e-23}.
\end{itemize}
All parts are optional,
except that there must be
either a whole part or a fraction part.

In addition to the above,
\fhicl treats the literal \fclcode{infinity}
as a number.
A sign
may optionally precede this literal.

\subsection{Interpretation}

Although the nomenclature varies,
programming languages typically distinguish
among real values, unsigned integer values, and signed integer values.
To the extent possible
in the underlying programming language,
each \fhicl binding
must, without error, interpret each \fhicl number:
\begin{itemize}
  \item as any kind of real value upon request;
  \item as an unsigned integer value upon request
        if the number can be equivalently rewritten
        using no more than seven digits;
  and
  \item as a signed integer value upon request
        if the number can be equivalently rewritten
        using only an optional \fclcode{-} sign
        followed by no more than seven digits.
\end{itemize}
\Needspace{0.67in}
\begin{fcllisting}[texcl,escapechar=`]
pi : 3.1415926   # strictly a real value
x  : 1.23e2      # same as 123: a signed, unsigned, or real value
y  : -.45600e+3  # same as -456: a signed or real value
z  : -infinity   # the most negative real value
\end{fcllisting}

\section{String values}

A \fhicl string is written as a sequence of characters
usually enclosed within matching quotation marks.
The quotation marks may be omitted,
but only if the string
contains no whitespace, punctuation, or other special characters. 
Strings that begin with a digit must be quoted
unless they have the form of a number
as described in the previous section.
\Needspace{1.34in}
\begin{fcllisting}[texcl,escapechar=`]
s1: a
s2: ab
s3: string
s4: "string"
s5: 'string'
s6: 12345
s7: "123abc"
s8: '123abc'
\end{fcllisting}

If the string is double-quoted,
\term{escaped characters} will be interpreted as follows:
  \fclcode{\\n} as a newline character,
  \fclcode{\\t} as a tab,
  \fclcode{\\'} as an apostrophe,
  \fclcode{\\"} as a double-quote,
and
  \fclcode{\\\\} as a (single) backslash.\footnote{%
                 The binding will produce a diagnostic
                 for any other escaped characters.%
                 }
If the string is single-quoted,
all characters are taken verbatim;
escaped characters have no special meaning.

\section{Complex values}

A \fhicl complex value is written
as two numbers
separated by a comma
and surrounded
by parentheses:
\Needspace{0.34in}
\begin{fcllisting}[texcl,escapechar=`]
c1: (1, 2)
c2: (1.23, -3.1415926)
\end{fcllisting}
Whitespace before and after each number is optional.

\section{Sequence values}

A \fhicl sequence starts with a left bracket
and ends with a right bracket.
These brackets surround
a comma-separated list
consisting of an arbitrary number
of \fhicl values.
Whitespace before and after each value is optional;
thus, the following three sequences are considered identical.
\Needspace{1.67in}
\begin{fcllisting}[texcl,escapechar=`]
q1: [ 1, 2, 3, 4 ]
q2: [ 1, 2
    , 3, 4
    ]
q3: [
      1,
      2,
      3,
      4
    ]
\end{fcllisting}

Note that \fhicl sequences 
may be \term{heterogeneous};
that is,
the \term{elements} (values in a sequence)
may be classified differently from each other.
For example,
some may be atomic while others are structured:
\Needspace{0.50in}
\begin{fcllisting}[texcl,escapechar=`]
q4: [ 1, (2, 3.14), "a b", nil ]  # 4 elements
q5: [ ]                           # 0 elements (empty)
q6: [ @local::q4, @local::q5 ]    # 2 elements
\end{fcllisting}
However, any given binding
can support such heterogeneity
only to the extent
that the underlying programming language supports it.\footnote{%
  This is rarely a restriction
  because, in practice,
  sequences tend overwhelmingly to be \term{homogeneous}.%
}

Individual elements may be identified
(in a reference, for example)
via a \term{subscript} (also known as an \term{index}) notation:
\Needspace{0.50in}
\begin{fcllisting}[texcl,escapechar=`]
const: [ 2.7182818, 3.1415926 ]
e    : @local::const[0]
pi   : @local::const[1]
\end{fcllisting}
The same notation can be used to override an element,
or even to extend a sequence with an additional element:
\Needspace{0.50in}
\begin{fcllisting}[texcl,escapechar=`]
fib: [ 0, 1, 1, oops, 3, 5 ]
fib[3] : 2
fib[6] : 8
\end{fcllisting}

\section{Table values}

A \fhicl table starts with a left brace
and ends with a right brace.
Much like a document,
the body of a \fhicl table
consists of name-value pairs.
Unlike a document,
a \fhicl table can't have a prolog.
The following document,
for example,
consists of a single name-value pair
whose value is a table consisting of three pairs:
\Needspace{0.67in}
\begin{fcllisting}[texcl,escapechar=`]
t : {  a : 5
       b : 6
       c : 7
    }
\end{fcllisting}

Individual entries within a table
may be identified
using a \term{member} notation.
For example,
continuing the previous example document:
\Needspace{0.17in}
\begin{fcllisting}[texcl,escapechar=`]
x : @local::t.c  # x's associated value is 7
\end{fcllisting}
The same notation can be used to override one of a table's values,
or even to \term{inject} an additional pair into a table:
\Needspace{0.34in}
\begin{fcllisting}[texcl,escapechar=`]
t.b : hi    # b's associated value is overridden with "hi"
t.d : 3.14  # table now gains a fourth pair
\end{fcllisting}


\chapter{Includes}

To deal with complexity,
it may be desirable
to assemble a larger \fhicl document
from several smaller parts,
with each fragment contained in its own file.
Such assembly is made possible
via \fhicl's \fclcode{#include} directive.

The syntax for the directive is very strict
in order to avoid possible confusion with a \fclcode{#}-introduced comment:
each \fclcode{#include} is on a line by itself,
with the \fclcode{#} in the first column.
There must be exactly one space following \fclcode{#include},
and then a double-quoted string identifying the file name\footnote{%
  It has become conventional
  to identify such files
  with the suffix \fclcode{.fcl}.%
}
of the target \fhicl document fragment.
\Needspace{0.34in}
\begin{fcllisting}[texcl,escapechar=`]
#include "filename1.fcl"
#include "/path/to/filename2.fcl"
\end{fcllisting}

A document may be composed
of as many such directives as desired.
The binding will replace each directive
with the document fragment contained in the corresponding named file.
A fragment may itself contain \fclcode{#include} directives.\footnote{%
  However, no fragment may include itself,
  even indirectly.%
}
Although such directives are most commonly found
at the start of a document,
they may appear wherever a user finds convenient.


\chapter{Prologs}

The purpose of a \fhicl \term{prolog}
(also known as a \term{prolog section})
is to provide name-value pairs
that can be referenced
later in the document.
Indeed,
pairs within a prolog
will not appear
in a parameter set
unless they are thusly referenced
outside the prolog.

A common use for a prolog
is to provide alternative values
from which to choose.
The following document uses a prolog in this fashion
such that only the 3-element sequence
will appear in the parameter set:
\Needspace{0.84in}
\begin{fcllisting}[texcl,escapechar=`]
BEGIN_PROLOG
  opt1: [0, 1, 2]
  opt2: [10, 11, 12, 13]
END_PROLOG
param: @local::opt1
\end{fcllisting}
The parameter set produced from the above document
is indistinguishable from that
produced from:
\Needspace{0.17in}
\begin{fcllisting}[texcl,escapechar=`]
param: [0, 1, 2]
\end{fcllisting}

A document may contain
as many or as few prolog sections as desired,
so long as each starts with \fclcode{BEGIN_PROLOG}
and ends with \fclcode{END_PROLOG}.
Only comments may precede
a prolog section;
no prolog sections are permitted
after a non-prolog name-value pair has appeared
in the document.

It is not possible to override
a prolog's name-value pair
from outside a prolog.
Pairs outside a prolog
that appear to override a prolog pair
are treated instead as new associations.


%\chapter{Examples}

%        \textbf{Example 1:}
%        \begin{verbatim}
%                #Document
%                x:5
%                y:4
%                z:3
%                tab1:{ a:1 b:2 c:3 }

%                #Parameter Set
%                { x:5 y:4 z:3 tab1:{ a:1 b:2 c:3 } }
%        \end{verbatim}

%        \textbf{Example 2:}
%        \begin{verbatim}
%                #Document
%                BEGIN\_PROLOG
%                x:1
%                y:2
%                END\_PROLOG
%                x: @local::x
%                y: 3
%                z: @local::y

%                #Parameter Set
%                { x:1 y:3 z:2 }
%        \end{verbatim}

%        \textbf{Example 3:}
%        \begin{verbatim}
%                #Document
%                BEGIN\_PROLOG
%                x:1
%                tab:{ a:1 b:[1,2,3] c:{ d:4 e:5 f:[6,7,8] } }
%                END\_PROLOG
%                tab1: @local::tab.c
%                tab2: @local::tab

%                #Parameter Set
%                { tab1: { d:4 e:5 f:[6,7,8] } tab2: { a:1 b:[1,2,3] c:{ d:4 e:5 f:[6,7,8] } } }
%        \end{verbatim}

%        \textbf{Example 4:}
%        \begin{verbatim}
%                #Document
%                BEGIN\_PROLOG
%                x:1
%                tab:{ a:1 b:[1,2,3] c:{ d:4 e:5 f:[6,7,8] } }
%                END\_PROLOG
%                tab.c : 5
%                tab1: @local::tab.c

%                #Parameter Set
%                { tab{ c:5 } tab1:{ d:4 e:5 f:[6,7,8] } }

%        \end{verbatim}


\chapter{Conclusion}
        This concludes the \fhicl Quick Start document.
        The definitions and examples given above will allow users to create their own \fhicl documents with ease,
        and should provide a quick reference for users who are familiar with the language but are in need of a primer.
\end{document}
