\documentclass[draftmode,draftwater]{memarticle}
\usepackage{fcllistings}

%------------------------------------------------------------
% Document title control. Change these as appropriate for your
% document.
%------------------------------------------------------------
\newcommand{\doctitle}{Quick Start Guide for FHiCL 3: \\
The \textbf{F}ermilab \textbf{Hi}erarchical \textbf{C}onfiguration \textbf{L}anguage}
\newcommand{\brieftitle}{FHiCL 3 Quick Start Guide}
\newcommand{\authors}{\href{mailto:paterno@fnal.gov}{Walter E.~Brown, Chris Green, Kyle Knoepfel, Jim Kowalkowski, Marc Paterno, Ryan Putz \\ Fermilab/SCD/SSA/SSI}}
\newcommand{\docversion}{1}

\usepackage{marvosym}%          to define \MVAt

\newcommand{\fhicl}{FHiCL\xspace}

% Symbols
\newcommand{\aterase}{\fclcode{@erase}\xspace}
\newcommand{\atlocal}{\fclcode{@local::}\xspace}
\newcommand{\atsequence}{\fclcode{@sequence::}\xspace}
\newcommand{\attable}{\fclcode{@table::}\xspace}
\newcommand{\atdb}{\fclcode{@db::}\xspace}
\newcommand{\atid}{\fclcode{@id::}\xspace}
\newcommand{\nil}{\fclcode{@nil}\xspace}
\newcommand{\protectError}{\fclcode{@protect_error:}\xspace}
\newcommand{\protectIgnore}{\fclcode{@protect_ignore:}\xspace}

%------------------------------------------------------------
\begin{document}
\maxtocdepth{chapter}

\topmatter

\chapter{Introduction}

The purpose of this document is to explain and demonstrate the syntax
and semantics of the Fermilab Hierarchical Configuration Language,
\fhicl%
\footnote{%
  The customary pronounciation is /fIkl/.%
}, so that users may become comfortable with its features and intended
use.

The end product of many years of experience with other configuration
languages and notation, \fhicl has been carefully designed to allow
its users to express, record, and retrieve sets of software parameters
used to \term{configure} (prepare for a particular purpose) a
program's execution.  We will use the term \term{parameter set} to
denote any specific collection of named values accessible to a user's
program while it is running; a \fhicl \term{document} is a textual
representation of such a parameter set.  Two or more \fhicl documents
are said to be \term{equivalent} if they lead to identical parameter
sets.

Depending on the programming language used to write the user program,
one of several software subsystems is used to produce a parameter set
from a given \fhicl document.  Each such subsystem is known as a
\term{binding} of the \fhicl specification to the programming
language, and provides for the analysis and interpretation of \fhicl
documents.

While it is up to each binding to specify how users interface to the
parameter sets produced by that binding, it is fundamental that users
be able to query a parameter set by providing it a name (in the form
of a string) in order to obtain the value corresponding to that name.
Further, a parameter set must be capable of producing a document that
is equivalent to the original document that gave rise to that
parameter set.


\chapter{Documents}

\section{Text}

A \fhicl document is simply a sequence of characters (\ie text),
structured as described below, and is commonly stored in a file whose
name is conventionally suffixed \fclcode{.fcl}.  For example,
\fclcode{my_config.fcl} might be the name of such a file.  Any
conventional text editor (\eg emacs, vi, nedit, \ldots) may be used to
create or to update a \fhicl file.%
\footnote{%
  Note that \fhicl documents need not be represented via any file.  A
  binding may, for example, obtain a \fhicl document from a database,
  via a string in a conventional programming language, or via any
  other mechanism that can denote simple text.  A binding is free to
  support an arbitrary number of document sources.%
}

\section{Name-value pairs}

A document consists principally of \term{name-value pairs}.%
\footnote{%
  A name-value pair is sometimes known as an \term{association},
  because parameter set lookup is designed to take a name and retrieve
  the associated value.  In this Guide we will usually prefer the
  simpler \term{pair} nomenclature.%
} There may be as many or as few such pairs as desired.%
\footnote{%
  A document consisting of no name-value pairs is said to be
  \term{empty}, as is the parameter set that a binding would generate
  from such a document.  Both the empty document and the empty
  corresponding parameter set are valid FHiCL constructs.%
} In each pair, a colon (\verb|:|) separates the \term{name} from the
corresponding \term{value}.%
\footnote{%
  In the context of the FHiCL language, the colon is referred to as
  the \term{standard binding operator}.  }

At least one blank, tab, or newline character (collectively known as
\term{whitespace}) must separate one pair from the next.  The
following document, consisting of three name-value pairs, uses the
minimum required whitespace: 
\Needspace{.17in}
\begin{fcllisting}[texcl,escapechar=`]
n:1 pi:3.14159 label:"horizontal axis"
\end{fcllisting}

\section{Optional whitespace}

Within a \fhicl document, additional whitespace may be used at the
discretion of the author.%
\footnote{%
  Such optional whitespace is commonly used to produce indentation or
  alignment.%
} A binding will ignore any such optional whitespace while producing a
parameter set from the document.

The following document is equivalent to the single-line document shown
above.  Consisting of the same three name-value pairs, this variation
employs extra whitespace to improve readability by (a) placing each
pair on an individual line and (b) aligning the values:
%
\Needspace{0.5in}
\begin{fcllisting}[texcl,escapechar=`]
n    : 1
pi   : 3.14159
label: "horizontal axis"
\end{fcllisting}

\chapter{Comments}

Document providers often wish to annotate the document's contents.
For example, it is common to provide a provenance giving the original
author's name and date, followed by a revision history.  Other
annotations include introductory overviews for each section of a
document, or even brief descriptions of individual name-value pairs.

\fhicl provides two ways to \term{introduce} (start) a comment:
\begin{itemize}
\item With a single \fclcode{#} character,%
  \footnote{%
    This notation is adopted from such scripting languages as bash,
    perl, and python.  The introductory \fclcode{#} character is known
    variously as a \term{pound sign}, \term{hash mark}, \term{sharp},
    or \term{octothorpe}%
    .}  or else
\item With two consecutive forward slashes (\fclcode{//}).\footnote{%
    This notation is adopted from such programming languages as BCPL
    and \cpp{}%
    .}
\end{itemize}
Either or both of these comment introductions may be used within any
\fhicl document.

The remainder of the line is the \term{body} of the comment, and
provides whatever information the author may desire.  The comment
implicitly terminates at the end of the line, although an author may
choose to continue his annotation onto any number of additional
comments on subsequent lines.

The following document illustrates the various forms that a \fhicl
comment may take.
%
\Needspace{0.67in}
\begin{fcllisting}[texcl,escapechar=`]
# This is a comment
// This is also a comment
foo : "bar"   # this is a comment "in the margin" ...
foo2: "bar2" // ... and so is this
\end{fcllisting}

\chapter{Names}

\section{Spelling}

The spelling rules for \fhicl names match the spelling rules for
identifiers in many programming languages:
\begin{itemize}
\item Each name begins with a letter or with a \fclcode{_}
  (underscore) character.
\item The name may be spelled with as many additional consecutive
  letters, underscores, or digits as desired.  No other characters
  (\eg punctuation or whitespace) may be embedded within a name.
\item Capitalization matters: the name \fclcode{X} is not the same
  name as the uncapitalized name \fclcode{x}.  Similarly, \fhicl
  treats the names \fclcode{Hello}, \fclcode{HELLO}, and
  \fclcode{hello} as three distinct, unrelated names.
\end{itemize}%

\section{Name reuse}

Consider the following document, noting especially the reuse of the
name \fclcode{a}.  When a binding processes this document, how many
name-value pairs will the resulting parameter set contain?
%
\Needspace{0.5in}
\begin{fcllisting}[texcl,escapechar=`]
a: 1
b: 2
a: 3
\end{fcllisting}
%
The answer is two.  \fhicl provides that, if a document has two
name-value pairs that have a name in common, the value in the later
pair \term{overrides} (supercedes) the earlier one.\footnote{%
  If more than two pairs have a name in common, the second pair
  overrides the first as described above until the third such pair is
  encountered.  Then the third pair overrides the second until a
  fourth pair is encountered, and so on.  In this way, the last pair
  using that name will ultimately override all the earlier ones with
  the same name.%
} Therefore, in the above example, when \fclcode{a} is looked up in
the parameter set, the associated value will be found to be
\fclcode{3}.  Section~\ref{sec:references} discusses how the values
associated with previously defined names can be used in other
locations of the document.

\chapter{Values}

\section{Classifications}

At a high level, each \fhicl value can be classified as either
\term{atomic} or \term{structured}.  A structured value can further be
categorized as a \term{sequence} or a \term{table}.  Each value,
therefore, falls under one of the following three \fhicl \term{categories}:

\begin{description}
\item [\textbf{atom}] A value that has no underlying structure.
\item [\textbf{sequence}] A collection of values that are not associated with any names.
\item [\textbf{table}] A collection of name-value pairs.\footnote{Also commonly referred to as a parameter set.}
\end{description}

We first discuss values of atomic type, and then discuss the sequence
and table.

\section{Atomic values}

\subsection{Boolean values}

The literals \fclcode{true} and \fclcode{false} correspond to the
customary truth values.  
%
\Needspace{0.17in}
\begin{fcllisting}[texcl,escapechar=`]
debug: true
\end{fcllisting}

\subsection{Numeric values}
%
%\subsection{Form}

As in most programming languages, a \fhicl number can have up to four
parts:
\begin{itemize}
\item The \term{sign} part consists of a single \fclcode{+} or
  \fclcode{-} character,
\item The \term{whole} part consists of a non-empty sequence of
  digits, such as \fclcode{0} or \fclcode{123}; any extraneous leading
  zeroes will be ignored by the binding.
\item The \term{fraction} part consists of a single \fclcode{.}
  character followed by a possibly empty sequence of digits; any
  extraneous trailing zeroes will be ignored by the binding.
\item The \term{exponent} part consists of a single \fclcode{e} or
  \fclcode{E} character, optionally followed by a sign, followed by a
  non-empty sequence of digits.  Examples include \fclcode{E5} and
  \fclcode{e-23}.
\end{itemize}
All parts are optional, but there must be at least one digit in either
the whole or the fraction part.

In addition to the above, \fhicl treats the literal \fclcode{infinity}
as a number.  A sign may optionally precede this literal.
%
%\subsection{Interpretation}
%
%Although the nomenclature varies,
%programming languages typically distinguish
%among real values, unsigned integer values, and signed integer values.
%To the extent possible
%in the underlying programming language,
%each \fhicl binding
%must, without error, interpret each \fhicl number:
%\begin{itemize}
%  \item as any kind of real value upon request;
%  \item as an unsigned integer value upon request
%        if the number can be equivalently rewritten
%        using no more than seven digits;
%  and
%  \item as a signed integer value upon request
%        if the number can be equivalently rewritten
%        using only an optional \fclcode{-} sign
%        followed by no more than seven digits.
%\end{itemize}
\Needspace{0.67in}
\begin{fcllisting}[texcl,escapechar=`]
pi: 3.1415926
x : 1.23e2
y : -.45600e+3
z : -infinity
\end{fcllisting}

A binding must take into account the mathematical value being
represented as well as any constraints imposed by the underlying
programming language.

\subsection{Complex values}

A \fhicl complex value is written as two numbers separated by a comma
and surrounded by parentheses:
% 
\Needspace{0.34in}
\begin{fcllisting}[texcl,escapechar=`]
c1: (1, 2)
c2: (1.23, -3.1415926)
\end{fcllisting}
%
Whitespace before and after each number is optional.

\subsection{String values}

A \fhicl string is written as a sequence of characters usually
enclosed within matching quotation marks.  The quotation marks may be
omitted, but only if the string contains no whitespace, punctuation,
or other special characters: 
%
\Needspace{1.17in}
\begin{fcllisting}[texcl,escapechar=`]
s1: a
s2: ab
s3: string
s4: "string"
s5: 'string'
s6: "123abc"
s7: '123abc'
\end{fcllisting}

If the string is double-quoted, \term{escaped characters} will be
interpreted as follows: \fclcode{\\n} as a newline character,
\fclcode{\\t} as a tab, \fclcode{\\'} as an apostrophe, \fclcode{\\"}
as a double-quote, and \fclcode{\\\\} as a (single)
backslash.\footnote{%
  The binding will produce a diagnostic error message for any other
  escaped characters.%
} If the string is single-quoted, all characters are taken verbatim;
escaped characters have no special meaning.

\subsection{\nil value}

The literal \nil serves as a placeholder value, distinct from all
other \fhicl values.  It is suitable for constructing a name-value
pair when no other \fhicl value will do.
%
\Needspace{0.17in}
\begin{fcllisting}[texcl,escapechar=`]
a: @nil
\end{fcllisting}

\section{Sequence values}

A \fhicl sequence starts with a left bracket and ends with a right
bracket.  These brackets surround a comma-separated list consisting of
an arbitrary number of \fhicl values.  Whitespace before and after
each value is optional; thus, the following three sequences are
considered identical.  
%
\Needspace{1.67in}
\begin{fcllisting}[texcl,escapechar=`]
q1: [ 1, 2, 3, 4 ]
q2: [ 1, 2
    , 3, 4
    ]
q3: [
   1,
   2,
   3,
   4
]
\end{fcllisting}

Note that \fhicl sequences may be \term{heterogeneous}; that is, the
\term{elements} (values in a sequence) may be classified differently
from each other.  For example, some may be numbers while others are
not: 
%
\Needspace{0.50in}
\begin{fcllisting}[texcl,escapechar=`]
q4: [ 1, (2, 3.14), "a b", @nil, true ]  # 5 elements
q5: [ ]                                  # 0 elements (empty)
q6: [ [12, 34], 5 ]                      # 2 elements
\end{fcllisting}
However, any given binding can support such heterogeneity only to the
extent that the underlying programming language supports
it.\footnote{%
  This is rarely a restriction because, in practice, sequences tend
  overwhelmingly to be \term{homogeneous}.%
}

% Individual elements may be identified (in a reference, for example)
% via a zero-based \term{subscript} (also known as an \term{index})
% notation:
% %
% \Needspace{0.50in}
% \begin{fcllisting}[texcl,escapechar=`]
% const: [ 2.7182818, 3.1415926 ]
% e    : @local::const[0]
% pi   : @local::const[1]
% \end{fcllisting}
A zero-based \term{subscript} (also known as an \term{index}) notation
can be used to override an individual element, or even to extend a
sequence with an additional element\footnote{%
  \fhicl sequences are \term{dense}: if a sequence contains $n$
  elements, their respective subscripts are always $0, 1, \ldots,
  n-1$.  Extending a sequence will implicitly insert \fclcode{@nil}
  values, if needed, to preserve this property.  }:
%
\Needspace{0.50in}
\begin{fcllisting}[texcl,escapechar=`]
fib   : [ @nil, 1, 1, "", 3, 5 ]  # 6 elements; heterogeneous
fib[0]: 0                         # @nil changed to '0'
fib[3]: 2                         # now a homogeneous sequence
fib[6]: 8                         # now 7 elements
fib[8]: 21                        # now 9 elements (fib[7] is @nil)
\end{fcllisting}

\section{Table values}

A \fhicl table starts with a left brace and ends with a right brace.
Much like a document, the body of a \fhicl table consists of
name-value pairs.  Unlike a document, a \fhicl table cannot have a
prolog.  The following document, for example, consists of a single
name-value pair whose value is a table consisting of three pairs:
%
\Needspace{0.67in}
\begin{fcllisting}[texcl,escapechar=`]
t: {
   a: 5
   b: 6
   c: { e: 2.718 }
}
\end{fcllisting}

The \term{member} notation can be used to override one of a table's values,
or even to \term{inject} an additional pair into a table:
%
\Needspace{0.34in}
\begin{fcllisting}[texcl,escapechar=`]
t.b: hi    # b's associated value is overridden with "hi"
t.d: 3.14  # table now gains a fourth pair
\end{fcllisting}

Note, however, that to override or inject any table values in this
way, the full qualification of the relevant name must be specified.
For example, in the following document
%
\Needspace{0.34in}
\begin{fcllisting}[texcl,escapechar=`]
t2: {
   c: { e: 2.718 }
   c.e : "energy"     # error
}
\end{fcllisting}
the name \fclcode{e} is only partially qualified when an attempt is
made to override its value.  The correct override syntax is to start
at the outer-most name and to use the member and subscript notation as
necessary:
%
\Needspace{0.34in}
\begin{fcllisting}[texcl,escapechar=`]
t2: { c: { e: 2.718 } }
t2.c.e : "energy"     # OK
\end{fcllisting}

\chapter{Names vs. keys}

It is helpful to distinguish between a \term{name} and a \term{key}.
In contrast to a name, a key can include member or subcript
notation.  Consider the following document:
%
\Needspace{0.34in}
\begin{fcllisting}[texcl,escapechar=`]
table: {
   sequence: [ { entry  : 1},
               { another: 2} ]
   atom: true
}
\end{fcllisting}
%
The list of names for this document is shown in Table~\ref{tab:names}.
Table~\ref{tab:keys} shows the complete list of keys associated with
the same document.  Note that each of the names listed in
Table~\ref{tab:names} is also a key listed in
Table~\ref{tab:keys}--i.e. names are a subset of keys.  This
distinction is important when considering references (see
section~\ref{sec:references}).
%
\begin{table}
  \caption{List of names for the above FHiCL document.}
  \begin{center}
    \begin{tabular}{ll} \hline\hline
      Name & Value type \\ \hline
      \fclcode{table} & table \\
      \fclcode{sequence} & sequence \\
      \fclcode{entry} & atom \\
      \fclcode{another} & atom \\
      \fclcode{atom} & atom \\ \hline\hline
    \end{tabular}
  \end{center}
  \label{tab:names}
\end{table}

\begin{table}
  \caption{List of keys associated with above document.  All keys that begin with the name \fclcode{table} are fully-qualified keys.}
  \begin{center}
    \begin{tabular}{ll} \hline\hline
      Key & Value type \\ \hline
      \fclcode{table} & table \\
      \fclcode{table.sequence} & sequence \\
      \fclcode{table.sequence[0]} & table \\
      \fclcode{table.sequence[0].entry} & atom \\
      \fclcode{table.sequence[1]} & table \\
      \fclcode{table.sequence[1].another} & atom \\
      \fclcode{table.atom} & atom \\ \hline
      \fclcode{sequence} & sequence \\
      \fclcode{sequence[0]} & table \\
      \fclcode{sequence[0].entry} & atom \\
      \fclcode{sequence[1]} & table \\
      \fclcode{sequence[1].another} & atom \\ \hline
      \fclcode{entry} & atom \\
      \fclcode{another} & atom \\
      \fclcode{atom} & atom \\ \hline\hline
    \end{tabular}
  \end{center}
  \label{tab:keys}
\end{table}

\chapter{References}
\label{sec:references}

Within some documents, a common scenario is to require that two (or
more) name-value pairs with distinct names nonetheless provide the
same value.  The most obvious approach is simply to duplicate the
value in each pair:
%
\Needspace{0.34in}
\begin{fcllisting}[texcl,escapechar=`]
m: 1
n: 1
\end{fcllisting}
%
However, as is true when writing programs, such duplication becomes
problematic over time because it is not obvious by inspection that
these two values are intended always to be identical.  As a result, if
a future update were to change the value associated with \fclcode{m},
the required corresponding change to \fclcode{n} could be easily
overlooked, especially if there were a great many intervening lines.

\section{Substitutions}

\subsection{\atlocal}

To help avoid such an unhappy scenario, \fhicl allows a value to refer
to a previously-provided value:
%
\Needspace{0.34in}
\begin{fcllisting}[texcl,escapechar=`]
m: 1
n: @local::m
\end{fcllisting}
%
A construction such as \fclcode{@local::m} is known as a \fhicl
\term{reference}.  Each reference consists of the prefix \atlocal
followed by a name (here, \fclcode{m}) from an earlier name-value
pair.%
\footnote{%
  The binding will report an error during processing if the name
  hasn't yet been seen.%
}
The following two documents are thus equivalent:

\begin{minipage}{0.48\textwidth}
\centering\textbf{Document 1}
  \begin{fcllisting}[texcl,escapechar=`]
a   : false
seq : [ a, b, c ]
tab : { d: e }

a1  : @local::a
a2  : @local::seq[1]
a3  : @local::tab.d
seq1: @local::seq
seq2: [ @local::seq, d ]
tab1: @local::tab
  \end{fcllisting}
\end{minipage}\hfill
\begin{minipage}{0.48\textwidth}
\centering\textbf{Document 2}
  \begin{fcllisting}[texcl,escapechar=`]
a   : false
seq : [ a, b, c ]
tab : { d: e }

a1  : false
a2  : b
a3  : e
seq1: [ a, b, c ]
seq2: [ [a, b, c], d ]
tab1: { d: e }
  \end{fcllisting}
\end{minipage}

\noindent Notice that the subscript and member notations can be used
for the \atlocal keyword in resolving references.  This is true, in
general, of any of the FHiCL references (see
section~\ref{sec:qualified}).

When a binding processes such a reference, the name is looked up among
the name-value pairs processed so far, and the corresponding value
\term{substituted} (used in place of the reference).  The value used
for the substitution is the value associated with the name at the time
the reference is parsed.  As a result, future revisions of the
configuration file which contain modifications to the value of the
first will automatically be propagated to the second, and the two
pairs will remain in sync.

In a single document, if the referenced name is given a new value,
that new value will be used for subsequent references; previously
processed references will not assume the new value.

\section{Splicing facilities}

\subsection{\attable}

The \attable keyword is used to allow the contents of a referenced
table to be spliced into the table in which it is invoked.  For
example, the following document:
%
\Needspace{0.34in}
\begin{fcllisting}[texcl,escapechar=`]
table1: {
   a: 1
   b: [2,3]
}
table2: {
   @table::table1
   c: 4
}
\end{fcllisting}
%
produces the same parameter set as:
%
\Needspace{0.34in}
\begin{fcllisting}[texcl,escapechar=`]
table1: {
   a: 1
   b: [2,3]
}
table2: {
   a: 1      # contents from
   b: [2,3]  # 'table1'
   c: 4
}
\end{fcllisting}

\subsection{\atsequence}

Similar to the \attable keyword, \atsequence is invoked to splice
sequence contents into already-existing sequences.  This document:
%
\Needspace{0.34in}
\begin{fcllisting}[texcl,escapechar=`]
seq1: [1,2,3]
seq2: [ @sequence::seq1, 4,5,6]
\end{fcllisting}
%
is equivalent to this one:
%
\Needspace{0.34in}
\begin{fcllisting}[texcl,escapechar=`]
seq1: [1,2,3]
seq2: [1,2,3, 4,5,6]
\end{fcllisting}

\section{References and fully-qualified keys}
\label{sec:qualified}

The above \atlocal, \attable, and \atsequence FHiCL-reference keywords
can be used only on fully qualified keys.  Consider the following
document:
%
\Needspace{0.34in}
\begin{fcllisting}[texcl,escapechar=`]
tab1: {
   tab2: {
      test: 4
      list: [6,5,4]
   }
}
\end{fcllisting}
%
To access any of the above values using either substitution or
splicing, the sequence of characters that follows \verb|::| must be a
fully qualified key.  The following invocations represents valid FHiCL
syntax (assuming the \fclcode{tab1} definition is visible):
%
\Needspace{0.34in}
\begin{fcllisting}[texcl,escapechar=`]
t2: {
   @table::tab1.tab2
   list: [ @sequence::tab1.tab2.list, 3,2,1 ]
}
a1: @local::tab1.tab2.list[0]
\end{fcllisting}
%
The above document is equivalent to:
%
\Needspace{0.34in}
\begin{fcllisting}[texcl,escapechar=`]
t2: {
   test: 4
   list: [6,5,4, 3,2,1]
}
a1: 6
\end{fcllisting}
%

Note that a fully-qualified key cannot be used until the definition of
the outer-most name is complete.  This means that this document:
%
\Needspace{0.34in}
\begin{fcllisting}[texcl,escapechar=`]
t1: {
   m1 : { setting: 1 }
   m2 : { setting: @local::t1.m1.setting }  # error
}
\end{fcllisting}
%
is not a valid FHiCL document because a reference to a \fclcode{t1}
member is invoked before the closing brace of \fclcode{t1} has been
reached.  A solution to this is to declare a name-value pair outside
of \fclcode{t1} that can be used inside of it:
%
\Needspace{0.34in}
\begin{fcllisting}[texcl,escapechar=`]
global_setting: 1
t1: {
   m1: { setting: @local::global_setting }
   m2: { setting: @local::global_setting }
}
\end{fcllisting}
%
The disadvantage in this case, however, is that an extra name
(\fclcode{global_setting}) has been introduced at outer-most scope to
support referencing within a local scope.  Since this name is meant to
merely support a single point of maintenance, and it is not meaningful
to what the document is trying to represent, its inclusion could be
unwanted.  In addition, if many such names are introduced, the
resulting parameter set can be unnecessarily large, and disentangling
the meaningful name-value pairs from those that are not meaningful can
be difficult.  This difficulty is resolved by the concept of prologs
(see section~\ref{sec:prologs}).

\chapter{Prologs}
\label{sec:prologs}

The purpose of a \fhicl \term{prolog} (also known as a \term{prolog
  section}) is to provide name-value pairs that can be referenced
later in the document without appearing in the final parameter set.

A common use for a prolog is to provide alternative values from which
to choose.  The following document uses a prolog in this fashion such
that only the 3-element sequence will appear in the parameter set:
%
\Needspace{0.84in}
\begin{fcllisting}[texcl,escapechar=`]
BEGIN_PROLOG
  opt1: [0, 1, 2]
  opt2: [10, 11, 12, 13]
END_PROLOG
param: @local::opt1
\end{fcllisting}
%
The parameter set produced from the above document is
indistinguishable from that produced from:
%
\Needspace{0.17in}
\begin{fcllisting}[texcl,escapechar=`]
param: [0, 1, 2]
\end{fcllisting}

A document may contain as many or as few prolog sections as desired,
so long as each starts with \fclcode{BEGIN_PROLOG} and ends with
\fclcode{END_PROLOG}.  No prolog may encompass another prolog; if
there is more than one, they must appear strictly sequentially.  Only
comments may precede a prolog section; no prolog sections are
permitted after a non-prolog name-value pair has appeared in the
document.

Name-value pairs defined in a prolog can be overridden from outside of
the prolog.  Consider the following document:
\Needspace{0.34in}
\begin{fcllisting}[texcl,escapechar=`]
BEGIN_PROLOG
a: { b: { c: 37 } }
END_PROLOG
a: { x: 12 }
\end{fcllisting}
%
The value associated with \fclcode{a} in the prolog is a table with an
additional table nested inside of it.  By reassigning the value of
\fclcode{a} outside of the prolog, the original prolog definition is
erased, and the parameter set generated from the above document is
equivalent to:
%
\Needspace{0.34in}
\begin{fcllisting}[texcl,escapechar=`]
a: { x: 12 }
\end{fcllisting}

\chapter{Includes}

To deal with complexity, it may be desirable to assemble a larger
\fhicl document from several smaller parts, with each fragment
contained in its own file.  Such assembly is made possible via
\fhicl's \fclcode{#include} directive.

The syntax for the directive is very strict in order to avoid possible
confusion with a \fclcode{#}-introduced comment: each
\fclcode{#include} is on a line by itself, with the \fclcode{#} in the
first column.  There must be exactly one space following
\fclcode{#include}, and then a double-quoted string identifying the
file name\footnote{%
  It has become conventional to identify such files with the suffix
  \fclcode{.fcl}.%
} of the target \fhicl document fragment.  
%
\Needspace{0.34in}
\begin{fcllisting}[texcl,escapechar=`]
#include "filename1.fcl"
#include "/path/to/filename2.fcl"
\end{fcllisting}

A document may be composed of as many such directives as desired.  The
binding will replace each directive with the document fragment
contained in the corresponding named file.\footnote{%
  If it is set, the environment variable \fclcode{FHICL_FILE_PATH}
  is consulted by the binding to locate a file so named.  The value of
  this variable is the usual colon-separated paths typified by the
  bash standard \fclcode{PATH} variable.%
} A fragment may itself contain \fclcode{#include}
directives.\footnote{%
  However, no fragment may include itself, even indirectly.%
} Although such directives are most commonly found at the start of a
document, they may appear wherever a user finds convenient.

\noindent It is \textbf{\textit{strongly}} recommended that only
prologs be placed in files that are \fclcode{#include}d.  This ensures
that users can most easily glean the structure of a given FHiCL
document under consideration.

\chapter{Additional facilities}

\section{\aterase}

Specifying the \aterase symbol as a value removes the
corresponding key from the table.  For example, in this configuration:
\Needspace{0.34in}
\begin{fcllisting}[texcl,escapechar=`]
a : {
   b1 : "some string"
}
a.b1 : @erase
\end{fcllisting}
the parameter \fclcode{b1} is removed, and the table \fclcode{a} is
empty.

\section{Modified binding operators and protection}

In addition to the standard binding operator (\verb|:|), there are two
additional binding operators---\protectIgnore and \protectError. These
bindings are single symbols in that no space is permitted between the
initial \fclcode{@} and the word, or between the word and the trailing
\fclcode{:}.  The three binding operators correspond to the following
protection levels, ordered by increasing priority level:

\begin{description}
\item [\textbf{None}] A value bound to a name using \verb|:| can be
  subsequently overridden.  This protection represents the lowest
  priority level of the three.
\item [\textbf{Ignore}] For a value bound using \protectIgnore,
  subsequent assignments to the specified name are ignored.
\item [\textbf{Error}] For a value bound using \protectError, a
  subsequent assignment attempt to the specified name is an error, for
  which an exception is thrown by the language binding, providing a
  diagnostic message to aid in problem resolution.
\end{description}

\subsection{Protection inheritance}

During assignment, a protection level is inherited from an enclosing
name if the nested parameter has no specified protection.  It is an
error, however, if the enclosed name has a specified protection level
of \term{ignore} when the enclosing name has a protection level of
\term{error}.  For example, the following FHiCL document:
%
\Needspace{0.34in}
\begin{fcllisting}[texcl,escapechar=`]
a @protect_ignore: { b: 13 }
\end{fcllisting}
is equivalent to:
\Needspace{0.34in}
\begin{fcllisting}[texcl,escapechar=`]
a @protect_ignore: {
   b @protect_ignore: 13
}
\end{fcllisting}
\Needspace{0.34in}
\begin{fcllisting}[texcl,escapechar=`]
a @protect_ignore: { b: 13 }
\end{fcllisting}

The following FHiCL document, however,
\Needspace{0.34in}
\begin{fcllisting}[texcl,escapechar=`]
a @protect_error: { b @protect_ignore: 13 }
\end{fcllisting}
is an error.

\subsection{Protection when using \aterase}

Use of \aterase on a name at a higher nesting level than that of a
protected name ignores the protection of the item.  For example:
%
\Needspace{0.34in}
\begin{fcllisting}[texcl,escapechar=`]
a1: { b: { x @protect_ignore: 7 } }
a2 @protect_ignore: { b: { x: 7 } }
a1: @erase
a2: @erase
\end{fcllisting}
%
is equivalent to
%
\Needspace{0.34in}
\begin{fcllisting}[texcl,escapechar=`]
a1: {}
a2: { b : { x : 7 } }
\end{fcllisting}
%
where the contents of \fclcode{a1} have been erased, and those of
\fclcode{a2} have been retained.

\subsection{Additional restrictions}

It is an error to use a modified binding operator for an assignment to
a name that already has a value.
%
\Needspace{0.34in}
\begin{fcllisting}[texcl,escapechar=`]
a: 2
a @protect_ignore: 3 # error
\end{fcllisting}
%

Similarly, a local or fully-qualified override shall honor protection,
whereas a nested replacement shall not. For example:
%
\Needspace{0.34in}
\begin{fcllisting}[texcl,escapechar=`]
a    : { b : { c @protect_error: 37 } }
d    : @local::a

a.b.c: 37             # error - protection honored
a    : 12             # OK    - protection overridden
d    : { b: { c: 43 } # OK    - protection overridden
\end{fcllisting}

In the case of a local override attempt, the protection is respected:
%
\Needspace{0.34in}
\begin{fcllisting}[texcl,escapechar=`]
a: {
   b: {
      c @protect_error: 37
      d: 31
      c: 43  #error - local override attempt
   }
}
\end{fcllisting}

\section{Miscellany}

\subsection{\atdb}
%
% The intent of \db is to allow users to

\subsection{\atid}

%\chapter{Examples}

%        \textbf{Example 1:}
%        \begin{verbatim}
%                #Document
%                x:5
%                y:4
%                z:3
%                tab1:{ a:1 b:2 c:3 }

%                #Parameter Set
%                { x:5 y:4 z:3 tab1:{ a:1 b:2 c:3 } }
%        \end{verbatim}

%        \textbf{Example 2:}
%        \begin{verbatim}
%                #Document
%                BEGIN\_PROLOG
%                x:1
%                y:2
%                END\_PROLOG
%                x: @local::x
%                y: 3
%                z: @local::y

%                #Parameter Set
%                { x:1 y:3 z:2 }
%        \end{verbatim}

%        \textbf{Example 3:}
%        \begin{verbatim}
%                #Document
%                BEGIN\_PROLOG
%                x:1
%                tab:{ a:1 b:[1,2,3] c:{ d:4 e:5 f:[6,7,8] } }
%                END\_PROLOG
%                tab1: @local::tab.c
%                tab2: @local::tab

%                #Parameter Set
%                { tab1: { d:4 e:5 f:[6,7,8] } tab2: { a:1 b:[1,2,3] c:{ d:4 e:5 f:[6,7,8] } } }
%        \end{verbatim}

%        \textbf{Example 4:}
%        \begin{verbatim}
%                #Document
%                BEGIN\_PROLOG
%                x:1
%                tab:{ a:1 b:[1,2,3] c:{ d:4 e:5 f:[6,7,8] } }
%                END\_PROLOG
%                tab.c : 5
%                tab1: @local::tab.c

%                #Parameter Set
%                { tab{ c:5 } tab1:{ d:4 e:5 f:[6,7,8] } }

%        \end{verbatim}


%\chapter{Conclusion}
%This concludes the \fhicl Quick Start document.
%The definitions and examples given above will allow users to create their own \fhicl documents with ease,
%and should provide a quick reference for users who are familiar with the language but are in need of a primer.

\end{document}
