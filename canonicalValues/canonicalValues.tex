\documentclass[12pt]{article}
\begin{document}
\thispagestyle{empty}
\section*{Purpose}
The templating, pretty printing and abstracting (check summing) operations
all need to use canonical representations.
Note, nothing in this specification requires any
particular intermediate storage format.
Using the canonical output format as the storage format is acceptable;
but, it specifically is {\em not} required.

The symbol, \textvisiblespace, represents a character of whitespace.

Unquoted numbers are in canonical representation format.

\section*{Input vs Canonical output formats}
\begin{center}
\begin{tabular}[ht]{||p{126pt}|p{72pt}|p{152pt}||}
\hline
\hline
Input & Canonical & Discussion \\
\hline
\hline
\verb|"nil"|, \verb|nil|, \verb|null| & \verb|nil| & \\
\hline
\verb|"null"| & \verb|"null"| &
{\raggedleft Not recognized as a nil value and remains a string.} \\
\hline
\verb|"true"|, \verb|true| & \verb|true| & \\
\hline
\verb|"false"|, \verb|false| & \verb|false| & \\
\hline
{\raggedleft \verb|"+infinity"|}, & \verb|+infinity| & \\
{\raggedleft \verb|"infinity"|}, & & \\
{\raggedleft \verb|+infinity|}, & & \\
{\raggedleft \verb|infinity|} & & \\
\hline
{\raggedleft \verb|"-infinity"|, \verb|-infinity|} & \verb|-infinity| & \\
\hline
\verb|"777"| & \verb|777| &
{\raggedleft String value recognized as a numeric value and becomes unquoted.}
\\
\hline
\verb|"007"|, \verb|007| & \verb|007| &
{\raggedleft Leading and trailing zeroes on numeric values are retained.} \\
\hline
\verb|"+007"|, \verb|+007| & \verb|+007| &
{\raggedleft Signs on numeric values, except for the infinities are preserved.}
\\
\hline
"\textvisiblespace\verb|(|\textvisiblespace\verb|01|\textvisiblespace\verb|,|\textvisiblespace\verb|2.0|\textvisiblespace\verb|)|\textvisiblespace" &
\verb|(01,2.0)| &
{\raggedleft
After stripping all whitespace, if recognizable as
a complex number per language grammar,
then the stripped text is the canonical format.} \\
\hline
\end{tabular}
\end{center}
\end{document}
